% !TEX program = pdflatex
\documentclass[11pt]{article}
\usepackage{times}
\usepackage{epsfig}
\usepackage{graphicx}
\usepackage{amsmath}
\usepackage{amssymb}
\usepackage{color}
\usepackage[ruled,linesnumbered]{algorithm2e}
\usepackage{tikz}
\usetikzlibrary{spy}
\usepackage{pdfpages}
\usepackage{listings}
 
% ref.
% https://tex.stackexchange.com/questions/87885/scalebar-overlay-and-tikz-spy
% https://tex.stackexchange.com/questions/389705/how-to-get-the-aspect-ratio-of-an-image-programmatically
% https://tex.stackexchange.com/a/2133
%  collected and written by Jun Liu
% School of Mathematics and Statistics, Northeast Normal University
% Copy Right Reserved
% version \alpha0.1
% If you have any suggestion, please email to liuj292@nenu.edu.cn
\usepackage{graphicx}
\usepackage{tikz}
\usepackage{subfigure}
\usetikzlibrary{positioning}
\usetikzlibrary{calc} % Only for Imagemagick workaround
\usepackage{ifthen}
\usepackage{xparse} % also loads expl3
 

 \ExplSyntaxOn
\NewDocumentCommand{\aspectratio}{smo}
 {% #2 is the image file
  \hbox_set:Nn \l_tmpa_box {\includegraphics{#2}}
  \IfNoValueTF{#3}
   {
    \__student_aspectratio:nn { \box_wd:N \l_tmpa_box } { \box_ht:N \l_tmpa_box }
   }
   {
    \IfBooleanTF{#1}{ \tl_gset:Nx } { \tl_set:Nx } #3
     {
      \__student_aspectratio:nn { \box_wd:N \l_tmpa_box } { \box_ht:N \l_tmpa_box }
     }
   }
 }

\cs_new:Nn \__student_aspectratio:nn
 {
  \fp_eval:n {round( #1 / #2 , 5)}
 }
\ExplSyntaxOff


  
 \ExplSyntaxOn
\NewDocumentCommand{\aspectratioNN}{smo}
 {% #2 is the image file
  \hbox_set:Nn \l_tmpa_box {\includegraphics{#2}}
  \IfNoValueTF{#3}
   {
    \__student_aspectratioNN:nn { \box_wd:N \l_tmpa_box } { \box_ht:N \l_tmpa_box }
   }
   {
    \IfBooleanTF{#1}{ \tl_gset:Nx } { \tl_set:Nx } #3
     {
      \__student_aspectratioNN:nn { \box_wd:N \l_tmpa_box } { \box_ht:N \l_tmpa_box }
     }
   }
 }

\cs_new:Nn \__student_aspectratioNN:nn
 {
  \fp_eval:n {round( #1)}
 }
\ExplSyntaxOff


 \ExplSyntaxOn
\NewDocumentCommand{\aspectratioHeight}{smo}
 {% #2 is the image file
  \hbox_set:Nn \l_tmpa_box {\includegraphics{#2}}
  \IfNoValueTF{#3}
   {
    \__student_aspectratioHeight:nn { \box_wd:N \l_tmpa_box } { \box_ht:N \l_tmpa_box }
   }
   {
    \IfBooleanTF{#1}{ \tl_gset:Nx } { \tl_set:Nx } #3
     {
      \__student_aspectratioHeight:nn { \box_wd:N \l_tmpa_box } { \box_ht:N \l_tmpa_box }
     }
   }
 }

\cs_new:Nn \__student_aspectratioHeight:nn
 {
  \fp_eval:n {round( #2)}
 }
\ExplSyntaxOff




 
 
 
\newcommand{\neworrenewcommand}[1]{\providecommand{#1}{}\renewcommand{#1}}

\newcommand{\zoomincludgraphic}[9]{
    \neworrenewcommand{\ffoo}[6]{
% #1 = width of displayed image
% #2 = image name
% #3 = x-coordinate of zoom-in portion in bottom left corner
% #4 = y-coordinate of zoom-in portion in bottom left corner
% #5 = x-coordinate of zoom-in portion in up right corner
% #6 = y-coordinate of zoom-in portion in up right corner
% #7 = actual size of the image in some units
% #8 = actual size of the scalebar in the same units than #7
% more details, please refer to DemoZoomInPortionExample.tex
\begin{tikzpicture}[x=#1\textwidth, y=#1\textwidth, font=\footnotesize]
%anchor 文字的位置,上南下北左东右西(与地图方位不一致)
% south 南;west 西; east 东; north 北
% above    right     left     below 稍有差别
\aspectratio{#2}[\imsizeratio] % the ratio of width and height of input image
\aspectratio{##6}[\imsizeration]
\aspectratioNN{#2}[\OrigImWid]
\aspectratioNN{##6}[\CropImWid]
\aspectratioHeight{##6}[\CropImHeight]
\aspectratioHeight{#2}[\OrigImHeight]
%\aspectratio*{example-image}[\imsizeratio]

  \node[anchor = south east, inner sep=0] (image) at (1,0) {\includegraphics[width=#1\textwidth]{#2}};
	    \coordinate (viewport lower left) at (#3,#4/\imsizeratio);  
	    \coordinate(viewport upper right) at (#5,#6/\imsizeratio);  
        \draw[##5, line width = ##4 pt] (viewport lower left) rectangle (viewport upper right);
        %画出内框(小框)
 
  %以下函数画放大框
     \pgfmathsetmacro{\multone}{#5-#3}
     \pgfmathsetmacro{\multtwo}{#6/\imsizeratio-#4/\imsizeratio}
     \pgfmathsetmacro{\coefscale}{#1*\CropImWid/\OrigImWid}
     %% 放大框在原图四个角的位置
     \ifthenelse{\equal{#9}{bottom_left} }{ 
	      \node[anchor=north, draw= ##3, inner sep=0pt, line width = ##2 pt, outer sep=0pt] (zoomPart) at
	      (#7*\CropImWid/\OrigImWid*0.5+##2/345, #7*\CropImHeight/\OrigImHeight/\imsizeratio+##2/345) { % 左下角
	      % 放大的方框 inner sep=0pt, 0pt表示放大的图片边缘离框的距离 
	       \scalebox{#7}{\tikz{	           
	         \node[anchor=south east, inner sep=0] at (1,1) {\includegraphics[width=\coefscale\textwidth]{##6}};
	         }}
	         };
         %% line connection
	   \ifthenelse{\equal{##1}{line_connection_on} }{ 
		  \draw[red, dashed] (viewport upper right|-viewport lower left) -- (zoomPart.north east); 
		  \draw[red, dashed] (viewport lower left) -- (zoomPart.north west);
		   }{}
	       
	 }{} 
	 
	 
     \ifthenelse{\equal{#9}{bottom_right} }{ 
	      \node[anchor=north, draw= ##3, inner sep=0pt, line width = ##2 pt, outer sep=0pt] (zoomPart) at
	      (1-#7*\CropImWid/\OrigImWid*0.5-##2/345,#7*\CropImHeight/\OrigImHeight/\imsizeratio+##2/345) { % 右下角
	       \scalebox{#7}{\tikz{	           
	         \node[anchor=south east, inner sep=0] at (1,1) {\includegraphics[width=\coefscale\textwidth]{##6}};
	         }}
	         };
         %% line connection         
		\ifthenelse{\equal{##1}{line_connection_on} }{ 
			  \draw[red, dashed] (viewport upper right|-viewport lower left) -- (zoomPart.south west); %red, dashed, ->
			  \draw[red, dashed] (viewport upper right) -- (zoomPart.north west);
		%	  \draw[red, dashed, ->] (viewport upper right|-viewport lower left) -- (zoomPart.north east);
		%	  \draw[red, dashed] (viewport lower left) -- (zoomPart.north west);
			   % (a |- b) has the x-coordinate of a and y-coordinate of b.
			   % (a -| b) has the y-coordinate of a, and x-coordinate of b
		   }{}
     }{}  
       
    \ifthenelse{\equal{#9}{up_right} }{  %-0.015
	      \node[anchor=north, draw= ##3, inner sep=0pt, line width = ##2 pt,outer sep=0pt] (zoomPart) at
	      (1-#7*\CropImWid/\OrigImWid*0.51-##2/345,1/\imsizeratio-##2/345) { % 右上角
	      % 放大的方框 inner sep=0pt, 0pt表示放大的图片边缘离框的距离 
	       \scalebox{#7}{\tikz{	           
	         \node[anchor=south east, inner sep=0] at (1,1) {\includegraphics[width=\coefscale\textwidth]{##6}};
	         }}
	         };
         %% line connection
	   \ifthenelse{\equal{##1}{line_connection_on} }{ 
		  \draw[red, dashed] (viewport lower left|-viewport upper right) -- (zoomPart.south west);
		  \draw[red, dashed] (viewport upper right) -- (zoomPart.south east);
		   }{}
     }{}   
       
       
       
     \ifthenelse{\equal{#9}{up_left} }{ 
	      \node[anchor=north, draw= ##3, inner sep=0pt, line width = ##2 pt,outer sep=0pt] (zoomPart) at
	      (#7*\CropImWid/\OrigImWid*0.51+##2/345,1/\imsizeratio-##2/345) { % 左上角
	      % 放大的方框 inner sep=0pt, 0pt表示放大的图片边缘离框的距离 
	       \scalebox{#7}{\tikz{	           
	         \node[anchor=south east, inner sep=0] at (1,1) {\includegraphics[width=\coefscale\textwidth]{##6}};
	         }}
	         };
         %% line connection
		   \ifthenelse{\equal{##1}{line_connection_on} }{ 
			  \draw[red, dashed] (viewport lower left|-viewport upper right) -- (zoomPart.south west);
			  \draw[red, dashed] (viewport upper right) -- (zoomPart.south east);
			   }{}
	     }{}
       
  
   
	 
	 
	%% help grid 
  	\ifthenelse{\equal{#8}{help_grid_on} }{ 
           \begin{scope}[
                x={(image.south east)},
                y={(image.north west)},
                font=\footnotesize,
                help lines,
                overlay
            ]
            
            \draw[help lines, xstep=.1,ystep=.1,overlay] (0,0) grid (1,1);
            \foreach \x in {0,1,...,9} { 
             %  \draw(\x/10,0) -- (\x/10,1);
                \node[anchor=north] at (\x/10,0) {0.\x}; %{\tiny0.\x}
            }
            \foreach \y in {0,1,...,9} {
             %   \draw(0,\y/10) -- (1,\y/10);                        
                \node[anchor=east] at (0,\y/10) {0.\y};
            }
        \end{scope}    
	}{}  
   
\end{tikzpicture}

    }
    \ffoo
}

 

 % !TEX program = pdflatex
\begin{document}
\date{October 1, 2021}
%%%%%%%%% TITLE
\title{An introduction to use imZoomConfig.tex}
\author{Jun Liu\\\texttt{liuj292@nenu.edu.cn}}
   

\maketitle

\begin{figure}[!ht]
\centering
	\subfigure[grid on, connection off, bottom left]{
	\zoomincludgraphic{0.45}{images/edgepreserving_cmp/lady_L0}{0.13}{0.65}{0.27}{0.81}{3}{help_grid_off}{bottom_left}{line_connection_off}{4}{blue}{2}{red}{images/crop_test} 
	}
	\subfigure[grid on, connection off, bottom right]{
	\zoomincludgraphic{0.45}{images/edgepreserving_cmp/lady_L0}{0.13}{0.65}{0.27}{0.81}{4}{help_grid_on}{bottom_right}{line_connection_on}{4}{blue}{2}{red}{images/crop_test} 
	}
	\subfigure[grid on, connection on, up left]{
	\zoomincludgraphic{0.45}{images/edgepreserving_cmp/lady_L0}{0.13}{0.65}{0.27}{0.81}{1}{help_grid_on}{up_left}{line_connection_on}{4}{blue}{2}{red}{images/crop_test} 
	}
	\subfigure[grid off, connection off, up right]{
	\zoomincludgraphic{0.45}{images/edgepreserving_cmp/lady_L0}{0.13}{0.65}{0.27}{0.81}{3}{help_grid_off}{up_right}{line_connection_off}{4}{blue}{2}{red}{images/crop_test} 
	}
	\caption{Illustration of imZoomConfig.tex}
	\label{illus}
\end{figure}

\section{Introduction}

In the literature of image processing, we usually need to display figures with some zoom-in portion for a clearer illustration. To this end, a toolkit named \texttt{imZoomConfig\_v2.tex} is provided. This toolkit is collected and written by Jun Liu, School of Mathematics and Statistics, Northeast Normal University, Copy Right Reserved. If you have any suggestion, please email to liuj292@nenu.edu.cn. The main function is \texttt{$\backslash$zoomincludgraphic} which has 14 input parameters. Suppose the inner clip portion is rectangle $\rm\bf{B}1$ and the corresponding zoom-in portion is rectangle $\rm\bf{B}2$.  The usage is as follows:

\texttt{$\backslash$zoomincludgraphic\{$\#$1\}\{$\#$2\}\{$\#$3\}\{$\#$4\}\{$\#$5\}\{$\#$6\}\{$\#$7\}\{$\#$8\}\{$\#$9\}}\\ \texttt{\{$\#\#$1\}\{$\#\#$2\}\{$\#\#$3\}\{$\#\#$4\}\{$\#\#$5\}} 

\begin{itemize}
	\item $\#$1: width of image, such as $0.3$ which means $0.3\backslash\rm{textwidth}$.
	\item $\#$2: image name including image location, such as $\rm{images\slash edgepreserving\_cmp\slash lady\_L0}$
	\item $\#$3: x-coordinate of the bottom-left corner of $\rm\bf{B}1$
	\item $\#$4: y-coordinate of the bottom-left corner of $\rm\bf{B}1$
	\item $\#$5: x-coordinate of the upper-right corner of $\rm\bf{B}1$
	\item $\#$6: y-coordinate of the upper-right corner of $\rm\bf{B}1$
	\item $\#$7: zoom-in ratio, such as 2, it means we will zoom-in 2 times larger of the selected portion
	\item $\#$8: to show grid or not. It helps to identify the coordinates. Two options: $\rm{help}\_grid\_on$ and $\rm{help}\_grid\_off$ 
	\item $\#$9: the location of zoom-in portion. It has four options: $\rm{up\_left}$, $\rm{up\_right}$, $\rm{bottom\_left}$, $\rm{bottom\_right}$
	\item $\#$10: two dashed lines connecting $\rm\bf{B}1$ and $\rm\bf{B}2$. It has two options: $\rm{line\_connection\_on}$ and $\rm{line\_connection\_off}$
	\item $\#$11: the borderline width of zoom-in portion $\rm\bf{B}2$, such as 3
	\item $\#$12: the borderline color of zoom-in portion $\rm\bf{B}2$, such as blue
	\item $\#$13: the borderline width of inner selected portion $\rm\bf{B}1$, such as 3
	\item $\#$14: the borderline color of inner selected portion $\rm\bf{B}1$, such as red
	\item $\#$15: the cropped portion name and its path. 
\end{itemize}
% 


\textcolor{red}{\bf{Caution !!!} The size of clip portion should be smaller than the size of the original image.}
% \section{Demo}
%The readers can use it similar as $\backslash \rm{includegraphics}$. To use this tool, the reader first need to put the $\texttt{imZoomConfig.tex}$ ahead of $\backslash$begin\{document\} using \texttt{$\backslash$input\{imZoomConfig.tex\}}
%
%The resulting figure is shown in Fig \ref{illus}. A demo is as follows:
%
%
%\begin{lstlisting}
%\documentclass[10pt]{article}
%\usepackage{times}
%\usepackage{epsfig}
%\usepackage{graphicx}
%\usepackage{amsmath}
%\usepackage{amssymb}
%\usepackage{tikz}
%\usetikzlibrary{spy}
%\usepackage{pdfpages}
%\usepackage{listings}
%\end{lstlisting}
%{\textcolor{red}{{$\backslash$input\{imZoomConfig.tex\}}}}
%\begin{lstlisting}
%\begin{document}
%
% 
%\title{An introduction to use imZoomConfig.tex}
%\maketitle
%
%
%\begin{figure}[!ht]
%\centering
%  \subfigure[]{
%  \zoomincludgraphic{0.45\textwidth}{images/edgepreserving_cmp/lady_L0}
%                    {0.2}{0.7}{0.3}{0.8}{3}{help_grid_on}{bottom_left}
%	            {line_connection_off}{4}{blue}{2}{red} 
%	}
%  \subfigure[]{
%  \zoomincludgraphic{0.45\textwidth}{images/edgepreserving_cmp/lady_L0}
%                    {0.3}{0.5}{0.4}{0.7}{2}{help_grid_on}{bottom_right}
%	            {line_connection_off}{4}{blue}{2}{red} 
%	}
%  \subfigure[]{
%  \zoomincludgraphic{0.45\textwidth}{images/edgepreserving_cmp/lady_L0}
%                    {0.6}{0.3}{0.7}{0.2}{3}{help_grid_on}{up_left}
%	            {line_connection_on}{4}{blue}{2}{red} 
%	}
%  \subfigure[]{
%  \zoomincludgraphic{0.45\textwidth}{images/edgepreserving_cmp/lady_L0}
%                    {0.3}{0.5}{0.4}{0.7}{2}{help_grid_off}{up_right}
%	            {line_connection_off}{4}{blue}{2}{red} 
%	}
%	\caption{testtest}
%\end{figure}
%\end{document}
%\end{lstlisting}
%

 
 


 
 
\end{document}
