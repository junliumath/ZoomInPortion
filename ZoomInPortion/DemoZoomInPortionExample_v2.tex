% !TEX program = pdflatex
\documentclass[11pt]{article}
\usepackage{times}
\usepackage{epsfig}
\usepackage{graphicx}
\usepackage{amsmath}
\usepackage{amssymb}
\usepackage{color}
\usepackage[ruled,linesnumbered]{algorithm2e}
\usepackage{tikz}
\usetikzlibrary{spy}
\usepackage{pdfpages}
\usepackage{listings}
 
\input{imZoomConfig_v2.tex}

 

 % !TEX program = pdflatex
\begin{document}
\date{October 1, 2021}
%%%%%%%%% TITLE
\title{An introduction to use imZoomConfig.tex}
\author{Jun Liu\\\texttt{liuj292@nenu.edu.cn}}
   

\maketitle

\begin{figure}[!ht]
\centering
	\subfigure[grid on, connection off, bottom left]{
	\zoomincludgraphic{0.45}{images/edgepreserving_cmp/lady_L0}{0.13}{0.65}{0.27}{0.81}{3}{help_grid_off}{bottom_left}{line_connection_off}{4}{blue}{2}{red}{images/crop_test} 
	}
	\subfigure[grid on, connection off, bottom right]{
	\zoomincludgraphic{0.45}{images/edgepreserving_cmp/lady_L0}{0.13}{0.65}{0.27}{0.81}{4}{help_grid_on}{bottom_right}{line_connection_on}{4}{blue}{2}{red}{images/crop_test} 
	}
	\subfigure[grid on, connection on, up left]{
	\zoomincludgraphic{0.45}{images/edgepreserving_cmp/lady_L0}{0.13}{0.65}{0.27}{0.81}{1}{help_grid_on}{up_left}{line_connection_on}{4}{blue}{2}{red}{images/crop_test} 
	}
	\subfigure[grid off, connection off, up right]{
	\zoomincludgraphic{0.45}{images/edgepreserving_cmp/lady_L0}{0.13}{0.65}{0.27}{0.81}{3}{help_grid_off}{up_right}{line_connection_off}{4}{blue}{2}{red}{images/crop_test} 
	}
	\caption{Illustration of imZoomConfig.tex}
	\label{illus}
\end{figure}

\section{Introduction}

In the literature of image processing, we usually need to display figures with some zoom-in portion for a clearer illustration. To this end, a toolkit named \texttt{imZoomConfig\_v2.tex} is provided. This toolkit is collected and written by Jun Liu, School of Mathematics and Statistics, Northeast Normal University, Copy Right Reserved. If you have any suggestion, please email to liuj292@nenu.edu.cn. The main function is \texttt{$\backslash$zoomincludgraphic} which has 14 input parameters. Suppose the inner clip portion is rectangle $\rm\bf{B}1$ and the corresponding zoom-in portion is rectangle $\rm\bf{B}2$.  The usage is as follows:

\texttt{$\backslash$zoomincludgraphic\{$\#$1\}\{$\#$2\}\{$\#$3\}\{$\#$4\}\{$\#$5\}\{$\#$6\}\{$\#$7\}\{$\#$8\}\{$\#$9\}}\\ \texttt{\{$\#\#$1\}\{$\#\#$2\}\{$\#\#$3\}\{$\#\#$4\}\{$\#\#$5\}} 

\begin{itemize}
	\item $\#$1: width of image, such as $0.3$ which means $0.3\backslash\rm{textwidth}$.
	\item $\#$2: image name including image location, such as $\rm{images\slash edgepreserving\_cmp\slash lady\_L0}$
	\item $\#$3: x-coordinate of the bottom-left corner of $\rm\bf{B}1$
	\item $\#$4: y-coordinate of the bottom-left corner of $\rm\bf{B}1$
	\item $\#$5: x-coordinate of the upper-right corner of $\rm\bf{B}1$
	\item $\#$6: y-coordinate of the upper-right corner of $\rm\bf{B}1$
	\item $\#$7: zoom-in ratio, such as 2, it means we will zoom-in 2 times larger of the selected portion
	\item $\#$8: to show grid or not. It helps to identify the coordinates. Two options: $\rm{help}\_grid\_on$ and $\rm{help}\_grid\_off$ 
	\item $\#$9: the location of zoom-in portion. It has four options: $\rm{up\_left}$, $\rm{up\_right}$, $\rm{bottom\_left}$, $\rm{bottom\_right}$
	\item $\#$10: two dashed lines connecting $\rm\bf{B}1$ and $\rm\bf{B}2$. It has two options: $\rm{line\_connection\_on}$ and $\rm{line\_connection\_off}$
	\item $\#$11: the borderline width of zoom-in portion $\rm\bf{B}2$, such as 3
	\item $\#$12: the borderline color of zoom-in portion $\rm\bf{B}2$, such as blue
	\item $\#$13: the borderline width of inner selected portion $\rm\bf{B}1$, such as 3
	\item $\#$14: the borderline color of inner selected portion $\rm\bf{B}1$, such as red
	\item $\#$15: the cropped portion name and its path. 
\end{itemize}
% 


\textcolor{red}{\bf{Caution !!!} The size of clip portion should be smaller than the size of the original image.}
% \section{Demo}
%The readers can use it similar as $\backslash \rm{includegraphics}$. To use this tool, the reader first need to put the $\texttt{imZoomConfig.tex}$ ahead of $\backslash$begin\{document\} using \texttt{$\backslash$input\{imZoomConfig.tex\}}
%
%The resulting figure is shown in Fig \ref{illus}. A demo is as follows:
%
%
%\begin{lstlisting}
%\documentclass[10pt]{article}
%\usepackage{times}
%\usepackage{epsfig}
%\usepackage{graphicx}
%\usepackage{amsmath}
%\usepackage{amssymb}
%\usepackage{tikz}
%\usetikzlibrary{spy}
%\usepackage{pdfpages}
%\usepackage{listings}
%\end{lstlisting}
%{\textcolor{red}{{$\backslash$input\{imZoomConfig.tex\}}}}
%\begin{lstlisting}
%\begin{document}
%
% 
%\title{An introduction to use imZoomConfig.tex}
%\maketitle
%
%
%\begin{figure}[!ht]
%\centering
%  \subfigure[]{
%  \zoomincludgraphic{0.45\textwidth}{images/edgepreserving_cmp/lady_L0}
%                    {0.2}{0.7}{0.3}{0.8}{3}{help_grid_on}{bottom_left}
%	            {line_connection_off}{4}{blue}{2}{red} 
%	}
%  \subfigure[]{
%  \zoomincludgraphic{0.45\textwidth}{images/edgepreserving_cmp/lady_L0}
%                    {0.3}{0.5}{0.4}{0.7}{2}{help_grid_on}{bottom_right}
%	            {line_connection_off}{4}{blue}{2}{red} 
%	}
%  \subfigure[]{
%  \zoomincludgraphic{0.45\textwidth}{images/edgepreserving_cmp/lady_L0}
%                    {0.6}{0.3}{0.7}{0.2}{3}{help_grid_on}{up_left}
%	            {line_connection_on}{4}{blue}{2}{red} 
%	}
%  \subfigure[]{
%  \zoomincludgraphic{0.45\textwidth}{images/edgepreserving_cmp/lady_L0}
%                    {0.3}{0.5}{0.4}{0.7}{2}{help_grid_off}{up_right}
%	            {line_connection_off}{4}{blue}{2}{red} 
%	}
%	\caption{testtest}
%\end{figure}
%\end{document}
%\end{lstlisting}
%

 
 


 
 
\end{document}
